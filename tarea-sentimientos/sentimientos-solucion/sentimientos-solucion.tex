\documentclass[14pt,a4paper]{report}
\usepackage[utf8]{inputenc}
%Idioma
\usepackage[spanish]{babel}
%Para expresiones matematicas
\usepackage{titlesec}
\usepackage{amsmath}
\usepackage{bm}
\usepackage{amsfonts}
\usepackage{amssymb}
\usepackage{amstext}
\usepackage{mathtools}
\usepackage{mathrsfs}
\usepackage{physics}
%Imagenes
\usepackage{graphicx}
\usepackage{color}
%Para dibujar
\usepackage{tikz}
\usetikzlibrary{arrows.meta}
\usetikzlibrary{decorations.markings}
\usetikzlibrary{babel,patterns,snakes}
\usetikzlibrary{shapes.callouts}
\usetikzlibrary{calc,patterns,angles,quotes}
\usetikzlibrary{shadings}
%Cajas con colores
\usepackage{tcolorbox}
\tcbuselibrary{theorems}
%Ejemplo de lo anterior
%\begin{equation}
% a x^2 + bx + c = 0 \rightarrow
%\tcboxmath[colback=magenta!25!white,colframe=magenta, title=Solución]
%{x = \frac{-b\pm\sqrt{b^2-4ac}}{2a}}  
%\end{equation}
%
%Estilo fancy
\usepackage{fancyhdr}
%Interlineado y margenes y poco mas
\usepackage{setspace}
\usepackage{parskip}
\usepackage{multicol}
\usepackage[left=2.5cm,right=2.5cm,top=2.5cm,bottom=2.5cm]{geometry}
%Empieza documento
\begin{document}
%Cajas comentadas 
\newcommand{\commentedbox}[2]{%
  \mbox{
    \begin{tabular}[t]{@{}c@{}}
    $\boxed{\displaystyle#1}$\\
    #2
    \end{tabular}%
  }%
}
%Definimos el estilo de las paginas
\pagestyle{fancy}
\lhead{\itshape Semestre 2023-1}
\rhead{Tarea 0}
\Large{\textit{Inteligencia Artificial} - Sentimientos}\\
\normalsize
Rodolfo Armando Jaramillo Ruiz\\
25 de Febrero de 2023\\
\subsection*{Construyendo la intuición}
\quad Se tiene el siguiente conjunto de mini-reseñas:
\begin{enumerate}
	\item $(-1)$ not good
	\item $(-1)$ pretty bad
	\item $(+1)$ good plot
	\item $(+1)$ pretty scenery
\end{enumerate}
A cada reseña se le asocia un vector de características $\phi(x)$ donde se asocia a cada palabra la cantidad de veces que aparece en la reseña. Por ejemplo:\\
\begin{equation*}
	\phi(x)=\{\text{not}:1,\text{good}:1\}
\end{equation*} 
Se usa la definición de pérdida de articulación
\begin{equation*}
	\text{Loss}_{\text{hinge}}=\text{max}\{0,1-\bm{w}\phi(x)y\}
\end{equation*}
Todo donde $x$ es el texto de la reseña, $y$ es la etiqueta correcta y $\bm{w}$ es el vector de pesos.\\
\textbf{a.} Usando el descenso de gradiente estocástico una por cada una de las reseñas en el orden en que se muestran














\end{document}