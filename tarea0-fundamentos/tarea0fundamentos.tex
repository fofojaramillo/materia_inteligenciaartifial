\documentclass[14pt,a4paper]{report}
\usepackage[utf8]{inputenc}
%Idioma
\usepackage[spanish]{babel}
%Para expresiones matematicas
\usepackage{titlesec}
\usepackage{amsmath}
\usepackage{amsfonts}
\usepackage{amssymb}
\usepackage{amstext}
\usepackage{mathtools}
\usepackage{mathrsfs}
\usepackage{physics}
%Imagenes
\usepackage{graphicx}
\usepackage{color}
%Para dibujar
\usepackage{tikz}
\usetikzlibrary{arrows.meta}
\usetikzlibrary{decorations.markings}
\usetikzlibrary{babel,patterns,snakes}
\usetikzlibrary{shapes.callouts}
\usetikzlibrary{calc,patterns,angles,quotes}
\usetikzlibrary{shadings}
%Cajas con colores
\usepackage{tcolorbox}
\tcbuselibrary{theorems}
%Ejemplo de lo anterior
%\begin{equation}
% a x^2 + bx + c = 0 \rightarrow
%\tcboxmath[colback=magenta!25!white,colframe=magenta, title=Solución]
%{x = \frac{-b\pm\sqrt{b^2-4ac}}{2a}}  
%\end{equation}
%
%Estilo fancy
\usepackage{fancyhdr}
%Interlineado y margenes y poco mas
\usepackage{setspace}
\usepackage{parskip}
\usepackage{multicol}
\usepackage[left=2.5cm,right=2.5cm,top=2.5cm,bottom=2.5cm]{geometry}
%Empieza documento
\begin{document}
%Cajas comentadas 
\newcommand{\commentedbox}[2]{%
  \mbox{
    \begin{tabular}[t]{@{}c@{}}
    $\boxed{\displaystyle#1}$\\
    #2
    \end{tabular}%
  }%
}
%Definimos el estilo de las paginas
\pagestyle{fancy}
\lhead{\itshape Semestre 2023-1}
\rhead{Tarea 0}
\Large{\textit{Inteligencia Artificial} - Fundamentos}\\
\normalsize
Rodolfo Armando Jaramillo Ruiz\\
27 de Enero de 2023\\
\subsection*{Optimización y probabilidad}

\textbf{a.}\quad Empiezo por derivar $f(\theta)$:
\begin{equation}
f'(\theta)=2\sum_{i=1}^{n}w_i(\theta-x_i)
\end{equation}
Posteriormente se iguala a cero y procedo a despejar $\theta$:
\begin{equation*}
2\sum_{i=1}^{n}w_i(\theta-x_i)=0
\end{equation*}
\begin{equation*}
\sum_{i=1}^{n}w_i\theta-\sum_{i=1}^{n}w_ix_i=0
\end{equation*}
\begin{equation*}
\theta\sum_{i=1}^{n}w_i=\sum_{i=1}^{n}w_ix_i
\end{equation*}  
\begin{equation}
\theta=\frac{\sum_{i=1}^{n}w_ix_i}{\sum_{i=1}^{n}w_i}
\end{equation} 
Obtuve el valor de $\theta$ que optimiza la la función, pero ¿cómo sé que es un valor mínimo? Usando el criterio de la segunda derivada, por lo que procedo a derivar nuevamente la ecuación (1).
\begin{equation}
f''(\theta)=2\sum_{i=1}^{n}w_i
\end{equation}
Inmediatamente se puede ver que la segunda derivada es siempre positiva, dado que $w_i$ son reales positivos. Por lo tanto, tomando el criterio de la segunda derivada, se concluye que el valor de $\theta$ obtenido en la ecuación (2) es un mínimo.\\
Este valor de $\theta$ obtenido tiene la forma de la expresión del centro de masa para una distribución discreta de masa, por tanto si hubiera alguna $w_i$ negativa, alejaría el valor del análogo al centro de masa, como si fuera una "masa negativa".\\
\textbf{b.}\quad Consideraciones:\\
\begin{equation}
\mathbf{x}=(x_1,...,x_d)\in\mathbb{R}^d
\end{equation}
\begin{equation}
f(\mathbf{x})=\max_{s\in [-1,1]}\Sigma_{i=1}^{d}sx_i
\end{equation}
\begin{equation}
g(\mathbf{x})=\Sigma_{i=1}^{d}\max_{s_i\in [-1,1]}s_ix_i
\end{equation}
Se ve que lo que hace la función $f$ es tomar la suma de $d$ reales y multiplicarla por un factor $s\in[-1,1]$ que lo haga máximo, mientras que $g$ hace lo mismo, pero no lo hace con la suma en sí, si no que multiplica a cada término por un $s_i$ haciendo a cada termino $s_ix_i$ lo más grande para cada término. Por lo tanto, $f$ es a lo más igual de grande que $g$ (como en el caso de $d=1$ donde $f=g$), es decir $f(\mathbf{x})\leq g(\mathbf{x})$.\\
\textbf{c.}\quad La función de recurrencia se contruye como sigue:\\
$$
	V(n)=0\quad\text{En el caso que n=1}
$$
$$
	V(n)=0\quad\text{En el caso que n=2}
$$
$$
	V(n)=V(n)-a\quad\text{En el caso que n=3}
$$
$$
	V(n)=V(n)\quad\text{En el caso que n=4}
$$
$$
	V(n)=V(n)\quad\text{En el caso que n=5}
$$
$$
	V(n)=V(n)+b\quad\text{En el caso que n=6}
$$
Por lo que la función completa sería:\\
\begin{equation}
	V(n)=4V(n)+b-a
\end{equation}
Por lo que solo queda calcular la esperanza como sigue:
$$
E[V(n)]=\frac{1}{6}(4V(n)+b-a)		
$$
\\
\textbf{d.}\quad Tengo que para la secuencia $\{S,A,A,A,S,A\}$ la probabilidad de obtenerla es la siguiente función: 
\begin{equation}
L(p)=(1-p)ppp(1-p)p=p^{4}(1-p)^{2}
\end{equation}
El valor de $p$ que maximiza la probabilidad de obtener esta secuencia se obtiene de la siguiente forma:\\
Primero; se puede ver que $L(p)$ es un polinomio de grado 6, por lo que la primera derivada es un polinomio de grado 5, que al igualarla a cero para optimizar $L(p)$ se obtendrá que hay que despejar una ecuación de quinto grado, lo cual es, por decir lo menos, complicado. Por lo tanto, me iré por el camino de optimizar la función $ln[L(p)]$ que, debido a que la función logaritmo natural es una función creciente, conserva los máximos y mínimos de la función de la función original.\\
Procedo:
\begin{equation}
\frac{d}{dp}\{ln[L(p)]\}=\frac{d}{dp}\{ln[p^4(1-p)^{2}]\}
\end{equation}
\begin{equation*}
\frac{d}{dp}\{ln[L(p)]\}=\frac{d}{dp}\{ln[p^4]+ln[(1-p)^{2}]\}
\end{equation*}
\begin{equation*}
\frac{d}{dp}\{ln[L(p)]\}=\frac{d}{dp}\{4ln[p]+2ln[(1-p)]\}
\end{equation*}
\begin{equation*}
\frac{d}{dp}\{ln[L(p)]\}=4\frac{d}{dp}ln[p]+2\frac{d}{dp}ln[(1-p)]
\end{equation*}
\begin{equation}
\frac{d}{dp}\{ln[L(p)]\}=\frac{4}{p}-\frac{2}{1-p}
\end{equation}
Se iguala la derivada a cero y se despeja $p$:\\
\begin{equation}
\frac{4}{p}-\frac{2}{1-p}=0
\end{equation}
\begin{equation*}
\frac{4(1-p)-2p}{p(1-p)}=0
\end{equation*}
\begin{equation*}
4(1-p)-2p=0
\end{equation*}
\begin{equation*}
4-4p-2p=0\rightarrow 4-6p=0
\end{equation*}
\begin{equation}
p=\frac{2}{3}
\end{equation}\\
Para saber si este valor es un máximo, tengo que verificar que la segunda derivada evaluada en este punto sea menor a cero:
\begin{equation}
\frac{d^2}{dp^2}\{ln[L(p)]\}=-\frac{4}{p^2}-\frac{2}{(1-p)^2}
\end{equation}
\begin{equation}
\frac{d^2}{dp^2}\{ln[L(p)]\}=-\left(\frac{4}{p^2}+\frac{2}{(1-p)^2}\right)
\end{equation}
Inmediatamente se ve que lo contenido entre paréntesis en la ecuación anterior es positivo, cada término es positivo sin importar el valor de $p$. Por lo tanto, $p=2/3$ maximiza la probabilidad de obtener la secuencia $\{S,A,A,A,S,A\}$. Significa que la moneda debe estar cargada para caer águila con probabilidad $2/3$ para que sea lo más probable obtener esa secuencia.\\

\textbf{e.}\quad	 Al ser que $P(A|B)=P(B|A)$ y que $P(X|Y)=P(X\cap Y)/P(X)$ se tiene que:
\begin{equation*}
\frac{P(A\cap B)}{P(B)}=\frac{P(A\cap B)}{P(A)}
\rightarrow\frac{1}{P(B)}=\frac{1}{P(A)}\rightarrow P(A)=P(B)
\end{equation*}
Ahora tenemos que la probabilidad de que ocurra A es igual a la probabilidad de que ocurra B. Considerando la probabilidad de la unión:
\begin{equation}
P(A\cup B)=P(A)+P(B)-P(A\cap B)=1/2
\end{equation}
\begin{equation*}
2P(A)-P(A\cap B)=1/2 \rightarrow 2P(A)=1/2+P(A\cap B)
\end{equation*}
\begin{equation}
P(A)=1/4+\frac{1}{2}P(A\cap B)
\end{equation}
Por lo tanto, sabiendo que $P(A\cap B)>0$:
\begin{equation}
P(A)>1/4
\end{equation}
\\
\textbf{f.}\quad La función\\
\begin{equation}
	f(\mathbf{w})=\left(\sum_{i=1}^{n}\sum_{j=1}^{n}(\mathbf{a}_{i}^{\top}\mathbf{w}-\mathbf{b}_{i}^{\top}\mathbf{w})^2\right)+\frac{\lambda}{2}||\mathbf{w}||_{2}^{2}
\end{equation}
Tiene un gradiente que se calcula como sigue:
\begin{equation*}
	\dv{w_{k}}f(w_{k})=\dv{w_{k}}\left[\left(\sum_{i=1}^{n}\sum_{j=1}^{n}(\mathbf{a}_{i}^{\top}w_{k}-\mathbf{b}_{i}^{\top}w_{k})^2\right)+\frac{\lambda}{2}||w_{k}||_{2}^{2}\right]
\end{equation*}
\begin{equation*}
	\dv{w_{k}}f(w_{k})=\left(\sum_{i=1}^{n}\sum_{j=1}^{n}2(\mathbf{a}_{i}^{\top}-\mathbf{b}_{i}^{\top})(\mathbf{a}_{i}^{\top}w_{k}-\mathbf{b}_{i}^{\top}w_{k})\right)+\lambda||w_{k}||_{2}
\end{equation*}
\begin{equation*}
	\dv{w_{k}}f(w_{k})=\left(2\sum_{i=1}^{n}\sum_{j=1}^{n}(\mathbf{a}_{i}^{\top}-\mathbf{b}_{i}^{\top})(\mathbf{a}_{i}^{\top}w_{k}-\mathbf{b}_{i}^{\top}w_{k})\right)+\lambda||w_{k}||_{2}
\end{equation*}
Este sería la forma de los elementos del gradiente:
\begin{equation}
	\nabla_{i} f(\mathbf{w})=\left(2\sum_{i=1}^{n}\sum_{j=1}^{n}(\mathbf{a}_{i}^{\top}-\mathbf{b}_{i}^{\top})(\mathbf{a}_{i}^{\top}w_{i}-\mathbf{b}_{i}^{\top}w_{i})\right)+\lambda||w_{i}||_{2}
\end{equation}
\subsection*{Complejidad}
\textbf{a.}\quad
Para empezar este problema podemos considerar el problema de un rectángulo en una cuadricula de $n\times n$, donde se tiene que recorrer cada vértice que conforma la cuadricula ($n^2$), en donde se va a incrementar la longitud y la altura ($n^2$), de forma que para este caso sería una complejidad $O(n^4)$.
Por lo que para cuatro cuadrados sería $O(n^16)$. 
\\
\textbf{b.}\quad
Bajo estas condiciones la función de recurrencia
\begin{equation*}
	V(i,j)=c(i,j)+\min\{V(i-1,j),V(i,j-1)\}
\end{equation*}
Para no calcular los mismos costos de utiliza una matriz que conserva los costos de cada nodo que ya fue calculado. El algoritmo queda como sigue:
$$
\textbf{function V:}
$$
$$
\textbf{if}\quad i==0\quad\textbf{or}\quad j==0\quad\textbf{then}
$$
$$\textbf{return}\quad 0
$$
$$
\textbf{end}
$$
$$
\textbf{if}\quad\textit{grid}[i][j]==cost\quad\textbf{then}
$$
$$
\textbf{return}\quad\textit{grid}[i][j]=c(i,j)+\min\{V(i-1,j),V(i,j-1)\}
$$
$$
\textbf{end}
$$
$$
\textbf{return}\quad grid[i][j]
$$
\\








\end{document}
